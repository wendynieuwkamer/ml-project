\documentclass{article}

\usepackage{booktabs} 
\usepackage[english]{babel}
\usepackage{graphicx}
\usepackage{amsmath}

\title {Project Report}
\date{21 December 2016}
\author{Elise Mol, Wendy Nieuwkamer, and Isobel Smith}

\begin{document}

\maketitle

\section{Problem}

After three months of learning about several machine learning techniques and classifying algorithms it was time to put our newly gained skills into practice. We decided to try our hand at one of the competitions hosted by Kaggle. The challenge we chose is Leaf Classification; the objective of the competition is to use binary leaf images and their extracted features to predict what species each leaf belongs to. The dataset which was provided includes around 1584 images of leaf specimens; these samples include 99 species. From every image three sets of features are extracted: a shape contiguous descriptor, an interior texture histogram and a fine-scale margin histogram. Every feature is represented as a 64-attribute vector. 


\section{Approach}
As the competition started at 30 August 2016, there were already quite some submissions made and kernels active. One of them stood out, namely "10 Classifier Showdown in Scikit-Learn". In this kernel
Jeff Delaney compared ten classifiers which are part of the python library Scikit-Learn. He used the default implementations of the following classifiers:

\begin{enumerate}
	\item K-neighbors
	\item Support Vector Machine
	\item Nu-Support Vector Machine
	\item Decision Tree
	\item Random Forest
	\item Ada Boost
	\item Gradient Boosting
	\item Gaussian Naive Bayes
	\item Linear Discriminant Analysis
	\item Quadratic Discriminant Analysis
\end{enumerate}

\newpage



\section{Choices made and justifications}
Which classifiers you chose ; why?  
Describe the relevant features of the classifiers.

\newpage

\newpage

\section{Solution}
How do we optimize them? k-fold cv, gridsearch
\newpage

\section{Evaluation of Approach and Solution} 
All the curves, the results

\newpage

\section{Sources}






















\end{document}