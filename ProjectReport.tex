\documentclass{article}

\usepackage{booktabs} 
\usepackage[english]{babel}
\usepackage{graphicx}
\usepackage{amsmath}
\usepackage{listings}
\usepackage{url}

\title {Project Report}
\date{21 December 2016}
\author{Elise Mol, Wendy Nieuwkamer, and Isobel Smith}

\begin{document}

\maketitle

\section{Problem}
	
	After three months of learning about several machine learning techniques and classifying algorithms it was time to put our newly gained skills into practice. We decided to try our hand at one of the competitions hosted by Kaggle. The challenge we chose is Leaf Classification; the objective of the competition is to use binary leaf images and their extracted features to predict what species each leaf belongs to. The dataset which was provided includes around 1584 images of leaf specimens; these samples include 99 species. From every image three sets of features are extracted: a shape contiguous descriptor, an interior texture histogram and a fine-scale margin histogram. Every feature is represented as a 64-attribute vector. 
	
	As the competition started at 30 August 2016, there were already quite some submissions made and kernels active. One of them stood out, namely "10 Classifier Showdown in Scikit-Learn". In this kernel Jeff Delaney compared ten classifiers which are part of the python library Scikit-Learn. The results of his experiment are shown in figure \ref{fig:showdown}. In his experiment Delaney used the default setup for the classifiers , combined with semi-random parameters. He notes that the performance of the classifiers could be improved by tuning the hyper-parameters. Thus, our research question is whether we can improve the accuracy found by Delaney by tuning the hyper-parameters of the classifiers.
	
	\begin{figure} 
		\includegraphics[width=\linewidth]{results-showdown.png}
		\caption{The 10 Classifier Showdown}
		\label{fig:showdown}
	\end{figure}
	
	As we did not have time to run experiments for all ten classifiers, we decided to look into five of them. We chose the K-neigbors classifier, decision tree classifier and Gaussian Naive Bayes as they had relatively decent accuracy at respectively 88\%, 65\%, and 57\%. Also, we had some background in these algorithms as they had been taught in our Machine Learning course. Additionally, we decided to choose either Ada boost or quadratic discriminant analysis as they could improve the most with found accuracies of 4.5\% and 2.5\%. 

\newpage
\section{The algorithms}
	Before we could start optimizing the hyper-parameters we had to research the algorithms we chose. In order to discover the most relevant information for our experiment we aimed to answer the following questions:
	
	\begin{itemize}
		\item How does the classifier work? 
		\item Which hyper-parameters exist for each classifier??
		\item Which hyper-parameters have the most effect on the classifier?
	\end{itemize}
	
	The answers to these questions are specific to the Scikit-Learn package.
	
	\subsection{K-neighbors}
		The k-neighbors algorithm classifies an object according to the k nearest objects from the training set. For a higher k noise might be filtered out, but class boundaries might become more vague and less accurate. Thus, it would be interesting to find out which value of k gives the highest accuracy for our dataset. There are two ways of computing the class from the neigbors. The first one is by giving all the neighbors equal weight, no matter how far or close they are.  The second is to give each neighbor a weight relative to its distance. The weighted version of the 
	
	\subsection{Decision Tree Classifier}
	
	\subsection{Gaussian Naive Bayes}
	
	\subsection{Ada Boost Classifier}
	
	\subsection{Quadratic Discriminant Analysis}

\section{Choices made and justifications}
	Which classifiers you chose ; why?  
	Describe the relevant features of the classifiers.

\newpage
\section{Hyper parameter optimization}

	Hyperparameter optimisation is one of the most important steps in machine learning \cite{bardenet}. The goal is to find the best hyperparameters for the given classifier, in order to optimise the loss function and to avoid overfitting \cite{Bergstra}. Some hyper parameter optimisation algorithms not only find the best hyperameters, but identify those that carry the most weight, \cite{Bergstra}.
	There are two hyperparameter optimisation algorithims built in to scikit-learn, grid search and randomised parameter optimisation, \cite{gridsearch}, as well as alternative methods such as model specific cross validation.  We chose grid search as we were recommended to use it, and it is one of the most widely used hyperparameter optimisation algorthims \cite{HyperparameterOptimisationWiki}.  
	
	\subsection{Grid Search}
	
		Grid Search optimises hyperparamters by iterating over all of the variables in a given range, and selecting the best combination to use in the chosen classifier. Grid search is guided by a perfomance metric, which is usually measured by cross-validation \cite{HyperparameterOptimisationWiki}. A disadvantage of grid search is that it can be time consuming and computationally expensive to run, \cite{HyperparameterOptimisationWiki}.
		As grid search was recommended to us, and is the most widely used way to optimize hyper parameters, we decided to use it in our project.
		
	\subsection{Grid Search in Scikit-Learn}
	
		Scikit-Learn has a built in function for Grid Search \cite{gridsearch}, which is defined below:
		
		
	\begin{lstlisting}
GridSearchCV(estimator, param_grid, scoring=None, fit_params=None, 
   n_jobs=1, iid=True, refit=True, cv=None, verbose=0, 
   pre_dispatch='2*n_jobs', error_score='raise',
   return_train_score=True)
	\end{lstlisting}
		
		The param \textunderscore grid is a dictionary, with the parameters you want to optimise as the keys and the range of parameters to try as the values. The grid search algorithm will then run the given classifier over the entire parameter grid in order to find the best possible combination. 
		
		Scoring allows you to choose to train for precision or recall. Precision is the fraction of retrieved instances that are relevant, whilst recall is the fraction of relevant instances that are retrieved.
		
		\begin{align*}
		Precision &= true positive / true positive + false positive\\
		Recall &= true positive / true positive + false negative 
		\end{align*}
		
		We decided to focus on precision, as we wanted to tune accurate classifiers. 
		
		The output of GridSearch is the a dict, with the names of the hyperparamter as keys, and the optimal result as values. 
		
		\begin{lstlisting}
{hyperparameter: optimal value, hyperparameter: optimal value}
		\end{lstlisting}

\newpage

\section{Solution}
	How do we optimize them? k-fold cv, gridsearch
\newpage

\section{Evaluation of Approach and Solution} 
	All the curves, the results

	The other hyperparamter optimisation function built in to  scikit-learn is randomized parameter optimisation \cite{gridsearch}. Instead of iterating over each potential combination of parameters, like grid search, it radomly selects these combinations. The advantage if this is that it can be less computationally expensive than grid search, and that adding parameters that do not influence the performance does not decrease the algorithms efficiency. The number of random samples that are tried can be specified 

\newpage
\section{Results}

The accuracy of classifiers with optimised hyperparameters were graphed against those without optimised hyperparameters:

\includegraphics[scale=0.7]{acc_class}

In each of the cases, grid search clearly improved the accuracy of the classifier, reaffirming the importance of hyperparameter optimisation in machine learning. 

AdaBoost was significantly improved by hyperparameter optimisation, from $4.56\%$ to [ask Elise for number]. This is due to the fact that the default parameters that were used in the 10 Classifier Showdown \cite{showdown} were inadequate for the leaf classification problem. 



\bibliographystyle{unsrt}
\bibliography{ProjectReport}


\section{Sources}


https://www.kaggle.com/jeffd23/leaf-classification/10-classifier-showdown-in-scikit-learn



















\end{document}